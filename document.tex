\documentclass[]{article}
 
\usepackage[T1]{fontenc}
\usepackage[utf8]{inputenc}
\usepackage[english, russian]{babel}
\usepackage{amsmath, amsthm}
\usepackage{amssymb}

\usepackage{tikz}
\usetikzlibrary{positioning}

\newtheorem{definition}{Определение}
\newtheorem{theorem}{Теорема}

%opening
\title{Математика}
\author{Денис Коломиец}

\begin{document}
	\tableofcontents
	\section{Общие математические сведения}
	\subsection{$\Rightarrow$}
	Знак $\Rightarrow$ означает <<следует>>. Например $A\Rightarrow B$ означает из $A$ следует $B$.
	\subsection{$\Leftrightarrow$}
	Знак $\Leftrightarrow$ означает <<равносильно>>. Например $A\Leftrightarrow B$ означает из $A$ равносильно $B$. Так же, если $A\Leftrightarrow B$ то это то же самое, $A\Rightarrow B$ и $B\Rightarrow A$.
	\subsection{Фигурная скобка $\{$}
	Запись
	\begin{equation*}
		\begin{cases}
			P_1
			\\
			P_2
		\end{cases}
	\end{equation*}
	означает, что условия $P_1$ и $P_2$ выполняются одновременно (\textbf{логическое И})
	\subsection{Квадратная скобка $[$}
	Запись
	\begin{equation*}
		\left[
		\begin{matrix}
			P_1
			\\
			P_2
		\end{matrix}
		\right.
	\end{equation*}
	означает, что хотя бы одно из условий $P_1$ или $P_2$ выполняются (\textbf{логическое ИЛИ})
	\subsection{Декартово произведение}
	\begin{definition}
		\textbf{Декартовым произведением} множеств $X$ и $Y$ называется множество всех упорядоченных пар $(x,y)$ где $x\in X, y\in Y$.
	\end{definition}
	\textbf{Обозначение}. $X\times Y$
	\\
	\\
	Декартово произведение можно обобщить для $n$ множеств, так, элементами множества $X_1\times X_2\times \ldots X_n$ будут \textbf{упорядоченные наборы} или \textbf{кортежи} $(x_1, x_2, \ldots, x_n)$ где $x_1\in X_1,x_2\in X_2, \ldots, x_n\in X_n$.
	
	\begin{definition}
	\textbf{Декартовой $n$-ой степенью} множества $X$ называется множество $X^n=X\times X\times \ldots X$ 
	\end{definition}
	\section{Функция}
	\subsection{Определение}
	\begin{definition}
		\textbf{Функцией, отображением} $f:X\rightarrow Y$ называется множество  упорядоченных пар $(x,y)$ где $x\in X, y\in Y$ такое, что 
		\begin{enumerate}
			\item для любого $x$ верно  $(x,y)\in f$; 
			\item для любых $x_1, x_2$ верно  $f(x_1)=f(x_2)$
			(функциональность)
		\end{enumerate}
		Множество $X$ называется \textbf{областью определения} $f$.
		Множество $Y$ называется \textbf{областью значений} $f$
	\end{definition}
	\textbf{Пример}. Пусть $X=\left\{1,2,3\right\}$ и $Y=\left\{a,b\right\}$, и мы хотим задать отображение $f:X\rightarrow Y$. В первую очередь, для каждого элемента из $X$ нужно указать значение из $Y$ (условие $1$ из определения). Допустим, для $f$ верно
	\begin{equation}
	\begin{cases}
	f(1)=a
	\\
	f(2)=a
	\\
	f(3)=b.
\end{cases}
	\end{equation}
Тогда, функция $f$ будет следующим множеством:
\begin{equation}
	f=\left\{(1,a),(2,a),(3,b)\right\}
\end{equation}

\textbf{Важно}. Если для любых $x_1, x_2$ верно $f(x_1)=f(x_2)$ (условие 2 из определения), то обратное, не обязательно верно, из того, что $f(1)=f(2)$ не следует, что $1=2$.
\\

\textbf{Важно}. Если для каждого $x\in X$ обязательно, чтобы $(x,y)\in f$ (условие 1 из определения), то необязательно, чтобы для каждого $y\in Y$ было верно $(x,y)\in f$.
\\

\textbf{Пример}. Пусть $X=\left\{1,2,3\right\}$ и $Y=\left\{a,b,c,d,e\right\}$, то множество 
\begin{equation}
	f=\left\{(1,a),(2,a),(3,b)\right\}
\end{equation}
является функцией.
\\
	
	Поскольку не для каждого множества можно выписать его элементы (любые бесконечные множества), возникает необходимость иначе задавать фукнции на таких множествах
\\
	
	\textbf{Пример}. Пусть $f:\mathbb{R}\rightarrow\mathbb{R}$ и $f(x)=2x$.
	Тогда функция $f$ является следующим множеством 
	\begin{equation}
		\left\{(x,2x)\text{ где }x\in\mathbb{R}\right\}
	\end{equation}
	то есть множеством всех пар $(x,2x)$ где $x\in\mathbb{R}$.
	
	\subsection{Биекции}
	
	Отображение переводят одни элементы в другие, по тому же самому принципу они и переводят множества в множества. Пусть
	
		\begin{equation}
		\begin{cases}
			f(1)=a
			\\
			f(2)=a
			\\
			f(3)=b.
		\end{cases}
	\end{equation}
	
	тогда $f(\left\{1,3\right\})=\left\{a,b\right\}$.
	\\
	
	Грубо говоря, если взять множество $A$ и применить к каждому его элементу функцию $f$ и результаты сложить в одном множестве $B$, получим образ множества $f(A)=B$
	\\
	
	Дадим строгое определение
	
	\begin{definition}
		Пусть $f:X\rightarrow Y$ и $A\subset X$. \textbf{Образом} $f(A)$ называется множество \textbf{всех} таких $y\in Y$ что существует $x\in X$ такой, что $$f(x)=y.$$
	\end{definition}
	
	\begin{definition}
		Функция $f:X\rightarrow Y$ называется \textbf{сюръекцией}, если $f(X) = Y$
	\end{definition}

	\begin{definition}
		Функция $f:X\rightarrow Y$ называется \textbf{инъекцией} или \textbf{однозначной}, если для любых $x_1, x_2$ верно, что если $f(x_1)=f(x_2)$ то $x_1=x_2$
	\end{definition}

	\begin{definition}
		Если функция \textbf{инъективна} и \textbf{сюръективна}, то она называется \textbf{биективной} или \textbf{взаимно-однозначной}
	\end{definition}
	
	\textbf{Примечание}. Биекция сопостовляет каждому $x$ ровно один $y$ (инъективность) и для каждого $y$ найдется $x$ (сюръективность), притом только один (функциональность).
	\\
	\\
	\subsubsection{Упражнение}
	\begin{enumerate}
		\item функциями
		\item сюръекциями
		\item инъекциями
		\item биекциями
	\end{enumerate} $$f:X\rightarrow Y$$ где $$X=\left\{1,2,3\right\}$$  $$Y=\left\{4,6,7\right\}$$
	
	\begin{enumerate}
		\item $\left\{(1,a),(2,a),(3,b)\right\}$ 
		\item $\left\{(1,4),(2,7)\right\}$ 
		\item $\left\{(1,4),(2,1),(3,7)\right\}$ 
		\item $\left\{(1,4),(2,7),(3,6)\right\}$
		\item $\left\{(1,4),(2,4),(3,4)\right\}$  
		\item $\left\{(1,4),(2,6),(3,7),(2,7)\right\}$  
	\end{enumerate}
	
	\subsubsection{Упражнение}
	Пусть \begin{gather*}
		f:\mathbb{R}^m\rightarrow\mathbb{R}^n
		\\
		g:\mathbb{R}^m\rightarrow\mathbb{R}^n
	\end{gather*}
	указать, какие из следующих функций существуют
	
	\begin{enumerate}
		\item $q(x)=af(x)+bg(x)$
		\item $q(x)=f(g(x))$
		\item $q(x)=af(x)+bh(x)$
		\item $q(x)=g(f(x))$
		\item $q(x)=h(f(x))$
	\end{enumerate}
	
	\section{Линейное пространство $\mathbb{R}^m$}
	
	Элементами $\mathbb{R}^m$ являются упорядоченные наборы $n$ вещественных чисел.
	
	Пусть \begin{gather*}
		\overline{x}=(x_1, x_2, \ldots, x_m)\in \mathbb{R}^m 
		\\
		\overline{y}=(y_1, y_2, \ldots, y_m)\in \mathbb{R}^m 
		\\
		a\in \mathbb{R}
	\end{gather*}
	\subsection{Определение}
	Введем следующие операции над вещественными числами:\begin{enumerate}
		\item Сумма: $$\overline{x}+\overline{y}=
		(x_1+y_1, x_2+y_2, \ldots, x_m+y_m)$$
		\item Умножение на число: $$a\overline{x}=
		(ax_1, ax_2, \ldots, ax_m)$$
	\end{enumerate}
	
	
	\subsection{Канонический базис и координаты}
	\begin{definition}
		\textbf{Линейной комбинацией} векторов 
		$$
		\overline{x_1}, \overline{x_2},\ldots, \overline{x_n}
		$$
		с коэффициентами 
		$$
		a_1,a_2,\ldots, a_n
		$$
		называется вектор $$\overline{x}=a_1\overline{x_1}+ a_2\overline{x_2},\ldots+ a_n\overline{x_n}$$.
	\end{definition}
	
	\begin{definition}
		\textbf{Каноническим базисом} в $\mathbb{R}^m$ называется упорядоченный набор векторов $(\overline{e_1}, \overline{e_2}, \ldots, \overline{e_m})$ где
		\begin{gather*}
			\overline{e_1} = (1,0,\ldots,0)
			\\
			\overline{e_2} = (0,1,\ldots,0)
			\\
			\ldots
			\\
			\overline{e_m} = (0,0,\ldots,1)
		\end{gather*}
	\end{definition}
	Таким образом, у вектора $\overline{e_i}$ в каноническом базисе на $i-$ой коориднате стоит $1$, а на остальных --- $0$.
	
	\begin{definition}
		Говорят, что вектор $\overline{x}$ разложен по каноническому базису, если он представлен в виде линейной комбинации
		$$\overline{x}=x_1\overline{e_1}+ x_2\overline{e_2},\ldots+ x_n\overline{e_n}.$$
	\end{definition}
	\textbf{Пример}.
	\begin{equation*}
		(5,3,2) = 5\overline{e_1}+3\overline{e_2}+2\overline{e_3}
	\end{equation*} 
	
	\section{Линейный оператор $f:\mathbb{R}^m\rightarrow\mathbb{R}^n$}
	\subsection{Определение}
	\begin{definition}
		Функция $f:\mathbb{R}^m\rightarrow\mathbb{R}^n$ называется \textbf{линейным оператором} если выполнены два условия \textbf{для любых} $\overline{x}, \overline{y}, a$
		\begin{enumerate}
			\item $$f(a\overline{x})=af(\overline{x})$$
			\item $$f(\overline{x}+\overline{y})=f(\overline{x})+f(\overline{y})$$
		\end{enumerate}
	\end{definition}
	
	\subsection{Общий вид линейных операторов}
	\subsubsection{Упражнение}
	\begin{theorem}
		\begin{equation}
			f(a_1\overline{x_1}+ a_2\overline{x_2},\ldots+ a_n\overline{x_n})
			=
			a_1f(\overline{x_1})+ a_2f(\overline{x_2})+\ldots+ a_nf(\overline{x_n})
		\end{equation}
	\end{theorem}
	\textbf{Доказательство}. Выполнить самостоятельно.
	\\
	\\
	Исследуем, как выглядят линейные операторы $f:\mathbb{R}^m\rightarrow\mathbb{R}^n$. Пусть $(\overline{e_1}, \overline{e_2}, \ldots, \overline{e_m})$ --- канонический базис, тогда 
		\begin{equation}\label{eq1}
			f(\overline{x})=
	f(x_1\overline{e_1}+ x_2\overline{e_2},\ldots+ x_n\overline{e_n})
	=
	x_1f(\overline{e_1})+ x_2f(\overline{e_2})+\ldots+ x_nf(\overline{e_n})
\end{equation}
	 Обзначим
	 \begin{equation}
	 	\overline{a_i}=
	 	f(\overline{e_i})=
	 	\begin{pmatrix}
	 		a_{1i}
	 		\\
	 		a_{2i}
	 		\\
	 		\ldots
	 		\\
			a_{ni}
	 	\end{pmatrix}
	 \end{equation}
	 $f(\overline{e_i})$ имеет $n$ координат, потому что $f$ переводит $m$-мерные векторы в $n$-мерные векторы. Строки идут от $1$ до $n$ а столбец имеет номер $i$. Преобразуем уравнение (\ref{eq1})
	 \begin{gather*}
	 	x_1f(\overline{e_1})+ x_2f(\overline{e_2})+\ldots+ x_nf(\overline{e_n})
	 	=
	 	\\
	 	x_1
	 	\begin{pmatrix}
	 		a_{11}
	 		\\
	 		a_{21}
	 		\\
	 		\ldots
	 		\\
	 		a_{n1}
	 	\end{pmatrix}
	 	+
	 	x_2
	 	\begin{pmatrix}
	 		a_{12}
	 		\\
	 		a_{22}
	 		\\
	 		\ldots
	 		\\
	 		a_{n2}
	 	\end{pmatrix}
	 	+
x_m
\begin{pmatrix}
	a_{1m}
	\\
	a_{2m}
	\\
	\ldots
	\\
	a_{nm}
\end{pmatrix}
	 \end{gather*}
	 Полученные векторы-столбцы пишут подряд, объединяя в матрицу. Далее, введем определение умножение матрицы на вектор так, чтобы 
	\begin{gather*}
		x_1
	\begin{pmatrix}
		a_{11}
		\\
		a_{21}
		\\
		\ldots
		\\
		a_{n1}
	\end{pmatrix}
	+
	x_2
	\begin{pmatrix}
		a_{12}
		\\
		a_{22}
		\\
		\ldots
		\\
		a_{n2}
	\end{pmatrix}
	+
	x_m
	\begin{pmatrix}
		a_{1m}
		\\
		a_{2m}
		\\
		\ldots
		\\
		a_{nm}
	\end{pmatrix}
	=
	\begin{pmatrix}
		a_{11} & a_{12} & \ldots & a_{1m}
		\\
		a_{21} & a_{22} & \ldots & a_{2m}
		\\
		\ldots & \ldots & \ddots & \vdots
		\\
		a_{n1} & a_{n2} & \ldots & a_{nm}
	\end{pmatrix}
	\begin{pmatrix}
		x_{1}
		\\
		x_{2}
		\\
		\ldots
		\\
		x_{m}
	\end{pmatrix}
	=A\overline{x}
\end{gather*}

\textbf{Важно}. Так же, матрицу $A$ можно записать на как набор векторов-столбцов
\begin{equation*}
	A=(\overline{a_1},\overline{a_1},\ldots,\overline{a_m})
\end{equation*}

Данные рассуждения оформим в качестве теоремы:
\begin{theorem}\label{th1}
	Линейные операторы $f:\mathbb{R}^m\rightarrow\mathbb{R}^n$ имеют вид $f(x)=Ax$, где $A$ --- матрица размерности $n\times m$ ($m$ столбцов и $n$ строк). Притом, $i-$ый вектор-столбец $\overline{a_i}$ матрицы $\overline{a_i}=f(\overline{e_i})$.
\end{theorem}

\subsection{Умножение матриц}
Пусть \begin{gather*}
	f:\mathbb{R}^m\rightarrow\mathbb{R}^n
	\\
	g:\mathbb{R}^n\rightarrow\mathbb{R}^p
\end{gather*}

Тогда $h(x)=g(f(x))$ --- линейный оператор.

\begin{enumerate}
	\item
	\begin{equation*}
		h(ax)=g(f(ax))=g(af(x))=ag(f(x))=ah(x)
	\end{equation*}
	\item
\begin{equation*}
	h(x+y)=g(f(x+y))=g(f(x)+f(y))=g(f(x))+g(f(y))=h(x)+h(y)
\end{equation*}
\end{enumerate}


Пусть $A=(\overline{a_1},\overline{a_2},\ldots,\overline{a_m})$ --- матрица $f$, $B=(\overline{b_1},\overline{b_2},\ldots,\overline{b_n})$ --- матрица $g$, $C=(\overline{c_1},\overline{c_2},\ldots,\overline{c_m})$ --- матрица $h$. Тогда
\begin{equation*}
	\overline{c_i}=h(\overline{e_i})=g(f(\overline{e_i}))=g(\overline{a_1})=B\overline{a_1}
\end{equation*}

\begin{definition}
	Полученная в результате матрица оператора $h$ называется \textbf{произведением} матрицы $B$ на $A$.
\end{definition}

\textbf{Важно}. Чтобы вычислить прозиведение $BA$ нужно последовательно применить матрицу $B$ к столбцам матрицы $A$ и последовательно объединить в матрицу.

\subsubsection{Упражнение}
Пусть \begin{gather*}
	f:\mathbb{R}^m\rightarrow\mathbb{R}^n
	\\
	g:\mathbb{R}^m\rightarrow\mathbb{R}^n
\end{gather*}
доказать что функция $h(x)=af(x)+bg(x)$ --- линейный оператор. Пусть $A=(\overline{a_1},\overline{a_2},\ldots,\overline{a_m})$ --- матрица $f$, $B=(\overline{b_1},\overline{b_2},\ldots,\overline{b_m})$ --- матрица $g$. Найти матрицу оператора $h$.

\section{Системы линейных уравнений (СЛАУ)}
\subsection{Определение}
\begin{definition}
Система следующего вида называется \textbf{системой линейных уравнений (СЛАУ)} относительно переменных $x_1,x_2,\ldots,x_m$. Данная система содержит $n$ уравнение и $m$ переменных.
\begin{equation*}
\begin{cases}
	a_{11}x_1 + a_{12}x_2 + \ldots + a_{1m}x_m=b_1
	\\
	a_{21}x_1 + a_{22}x_2 + \ldots + a_{2m}x_m=b_2
	\\
	\ldots
	\\
	a_{n1}x_1 + a_{n2}x_2 + \ldots + a_{nm}x_m=b_n
\end{cases}
\end{equation*}
\end{definition}
\subsection{Матричное представление}

\begin{gather*}
	\begin{cases}
		a_{11}x_1 + a_{12}x_2 + \ldots + a_{1m}x_m=b_1
		\\
		a_{21}x_1 + a_{22}x_2 + \ldots + a_{2m}x_m=b_2
		\\
		\ldots
		\\
		a_{n1}x_1 + a_{n2}x_2 + \ldots + a_{nm}x_m=b_n
	\end{cases}
	\Leftrightarrow
	\begin{pmatrix}
	a_{11}x_1 + a_{12}x_2 + \ldots + a_{1m}x_m
	\\
	a_{21}x_1 + a_{22}x_2 + \ldots + a_{2m}x_m
	\\
	\ldots
	\\
	a_{n1}x_1 + a_{n2}x_2 + \ldots + a_{nm}x_m
\end{pmatrix}
=
	\begin{pmatrix}
	b_1
	\\
	b_2
	\\
	\ldots
	\\
	b_n
\end{pmatrix}
\Leftrightarrow
\\
\Leftrightarrow
	\begin{pmatrix}
	a_{11} & a_{12} & \ldots & a_{1m}
	\\
	a_{21} & a_{22} & \ldots & a_{2m}
	\\
	\ldots & \ldots & \ddots & \vdots
	\\
	a_{n1} & a_{n2} & \ldots & a_{nm}
\end{pmatrix}
\begin{pmatrix}
	x_{1}
	\\
	x_{2}
	\\
	\ldots
	\\
	x_{m}
\end{pmatrix}
=
\begin{pmatrix}
	b_1
	\\
	b_2
	\\
	\ldots
	\\
	b_n
\end{pmatrix}
\Leftrightarrow
A\overline{x}=\overline{b}
\end{gather*}

Ранее установлено, что умножение матрицы на вектор --- есть линейный оператор $f:\mathbb{R}^m\rightarrow\mathbb{R}^n$. Отсюда получим теорему

\begin{theorem}
Решать систему линейных уравнений --- то же самое, что решать уравнение $f(\overline{x})=\overline{b}$, где $\overline{x}\in \mathbb{R}^m,b\in\mathbb{R}^n$ и $f:\mathbb{R}^m\rightarrow\mathbb{R}^n$ --- линейные оператор.
\end{theorem}

Рассмотрим СЛАУ

\begin{gather*}
	\begin{pmatrix}
		a_{11} & a_{12} & \ldots & a_{1m}
		\\
		a_{21} & a_{22} & \ldots & a_{2m}
		\\
		\ldots & \ldots & \ddots & \vdots
		\\
		a_{n1} & a_{n2} & \ldots & a_{nm}
	\end{pmatrix}
	\begin{pmatrix}
		x_{1}
		\\
		x_{2}
		\\
		\ldots
		\\
		x_{m}
	\end{pmatrix}
	=
	\begin{pmatrix}
		b_1
		\\
		b_2
		\\
		\ldots
		\\
		b_n
	\end{pmatrix}
\end{gather*}

Ей в соответсвтие можно сопоставить матрицу

\begin{gather*}
	\left(
	\begin{array}{llll|l}
		a_{11} & a_{12} & \ldots & a_{1m} & b_{1}
		\\
		a_{21} & a_{22} & \ldots & a_{2m} & b_{2} 
		\\
		\ldots & \ldots & \ddots & \vdots & \vdots
		\\
		a_{n1} & a_{n2} & \ldots & a_{nm} & b_{n}
	\end{array}
	\right)
\end{gather*}

здесь вертикальная черта не несет в себе никакого математического смысла и просто нужна, чтобы визуально отделить матрицу СЛАУ и столбец-решение.

\begin{definition}
	Пусть дана СЛАУ
	\begin{equation*}
			\begin{cases}
			a_{11}x_1 + a_{12}x_2 + \ldots + a_{1m}x_m=b_1
			\\
			a_{21}x_1 + a_{22}x_2 + \ldots + a_{2m}x_m=b_2
			\\
			\ldots
			\\
			a_{n1}x_1 + a_{n2}x_2 + \ldots + a_{nm}x_m=b_n
		\end{cases}
	\end{equation*}
	\textbf{Главная матрица системы}
	\begin{equation*}
			\begin{pmatrix}
			a_{11} & a_{12} & \ldots & a_{1m}
			\\
			a_{21} & a_{22} & \ldots & a_{2m}
			\\
			\ldots & \ldots & \ddots & \vdots
			\\
			a_{n1} & a_{n2} & \ldots & a_{nm}
		\end{pmatrix}
	\end{equation*}
	\textbf{Дополненная матрица системы}
	\begin{equation*}
	\left(
\begin{array}{llll|l}
	a_{11} & a_{12} & \ldots & a_{1m} & b_{1}
	\\
	a_{21} & a_{22} & \ldots & a_{2m} & b_{2} 
	\\
	\ldots & \ldots & \ddots & \vdots & \vdots
	\\
	a_{n1} & a_{n2} & \ldots & a_{nm} & b_{n}
\end{array}
\right)
\end{equation*}
\end{definition}

\subsection{Метод гаусса}

\begin{definition}
	Следующие действия с матрицей называются \textbf{элементарными преобразованиями матрицы}:
	\begin{enumerate}
		\item добавление к строке матрицы другой строки;
		\item умножение строки матрицы на число, отличное от нуля.
	\end{enumerate}
	если матрица $B$ получена из $A$ элементарными преобразованиями, то пишут $A\rightarrow B$.
\end{definition}

\begin{theorem}
	Пусть $A,B$ --- матрицы и $A\rightarrow B$. Тогда системы $A\overline{x}$ и $B\overline{x}$ равносильны.
\end{theorem}

\textbf{Пример}. Пусть дана система. 
\begin{gather*}
	\begin{cases}
		x+2y=3
		\\
		2x+y=6
	\end{cases}
\end{gather*}
Прибавим к первой строке вторую (это равносильное преобразование)
\begin{gather*}
	\begin{cases}
		x+2y=3
		\\
		2x+y=6
	\end{cases}
	\Leftrightarrow 
	\begin{cases}
		3x+3y=9
		\\
		2x+y=6
	\end{cases}
\end{gather*}
Перепишем эти рассуждения в матричном виде
\begin{gather*}
	\left(
	\begin{array}{ll|l}
		1 & 2 & 3
		\\
		2 & 1 & 6
	\end{array}
	\right)
	\rightarrow
	\left(
	\begin{array}{ll|l}
		3 & 3 & 9
		\\
		2 & 1 & 6
	\end{array}
	\right)
\end{gather*}

\begin{definition}
	содержимое...
\end{definition}

Преобразуем эту матрицу так, чтобы \textbf{главная матрица системы} стала единичной.

\begin{gather*}
	\left(
	\begin{array}{ll|l}
		1 & 2 & 3
		\\
		2 & 1 & 6
	\end{array}
	\right)
	\rightarrow
	\left(
\begin{array}{ll|l}
	1 & 2 & 3
	\\
	0 & -3 & 0
\end{array}
\right)
\rightarrow
	\left(
\begin{array}{ll|l}
	3 & 6 & 9
	\\
	0 & -3 & 0
\end{array}
\right)
\rightarrow
\\
\rightarrow
\left(
\begin{array}{ll|l}
	3 & 0 & 9
	\\
	0 & -3 & 0
\end{array}
\right)
\rightarrow
\left(
\begin{array}{ll|l}
	1 & 0 & 3
	\\
	0 & 1 & 0
\end{array}
\right)
\end{gather*}

По итогу, получаем следующее

\begin{gather*}
\begin{pmatrix}
	1 & 0 
	\\
	0 & 1 
\end{pmatrix}
\begin{pmatrix}
	x
	\\
	y 
\end{pmatrix}
=
\begin{pmatrix}
	3
	\\
	0 
\end{pmatrix}
\Leftrightarrow
\begin{pmatrix}
	x
	\\
	y 
\end{pmatrix}
=
\begin{pmatrix}
	3
	\\
	0 
\end{pmatrix}
\Leftrightarrow
\begin{cases}
	x=3
	\\
	y=0
\end{cases}
\end{gather*}


\textbf{Важно}. Таким образом, суть метода Гаусса заключается в том, чтобы элементарными преобразованиями сделать из главной матрицы системы единичную.

\textbf{Важно}. Последовательность элементарных преобразований \textbf{зависит} только от \textbf{главной матрицы} и \textbf{ек зависит} от столбца-решения.

\subsubsection{Упражнение}
Решить СЛАУ \textbf{строго теми же} элементарными преобразованиями, что из примера
\begin{equation*}
		\left(
	\begin{array}{ll|l}
		1 & 2 & 2
		\\
		2 & 1 & 2
	\end{array}
	\right)
\end{equation*}

\section{Матричные уравнения}
\end{document}
